\documentclass[11pt]{article}
\usepackage{fullpage}
\usepackage{amsmath,amssymb,amsfonts,float}
\renewcommand\vec[1]{\mathbf{#1}}
\usepackage{listings}
\usepackage[table]{xcolor}
\usepackage{framed,comment}

\newcommand\red[1]{\textcolor{red}{#1}}
\specialcomment{solution}{\bigskip\begin{leftbar}\par\noindent\textbf{Solution.} }{\end{leftbar} }
\begin{document}

\begin{tabular}{l}
\textbf{CSCI 5654-Fall16}: Assignment \#2 \\

\textbf{Robert Werthman} \phantom{Supercalifragilisticexpialidocius Smith}\\
\hline
\\[10pt]
\end{tabular}

\noindent
\textbf{P1.}


\noindent (A)
\begin{center}
\begin{tabular}{|c|c|}
\hline
Entering variables & Leaving variables\\
\hline
$x_2$ & $w_1\ \text{or}\ w_5$ \\
\hline
$x_5$ & $w_5$\\
\hline
$x_6$ & $w_2\ \text{or}\ x_3$\\
\hline
$w_3$ & $w_2\ \text{or}\ x_3$\\
\hline
\end{tabular}
\end{center}

\medskip

\noindent\textbf{(B)}\\ 
To find the corresponding solutions, we need to solve for $x_2$ in the $w_5$ row of the dictionary:
$$
x_2 = 2 - \frac{x_1}{2} - 2x_5 + \frac{w_4}{2} - \frac{w_5}{2}
$$
This equation is then plugged into all the places $x_2$ is a nonbasic variable.
\begin{center}
\begin{tabular}{|c|c|}
\hline
Basic variables & Solutions\\
\hline
$w_1$ & 0\\ \hline
$w_2$ & 2\\ \hline
$x_3$ & 4\\ \hline
$x_4$ & 1\\ \hline
$x_2$ & 2\\ \hline
$z$ & 15\\ \hline
\end{tabular}
\hspace{1cm}
\begin{tabular}{|c|c|}
\hline
Nonbasic variables\\
\hline
$x_1$\\ \hline
$w_5$\\ \hline
$x_5$\\ \hline
$x_6$\\ \hline
$w_3$\\ \hline
$w_4$\\ \hline
\end{tabular}
\end{center}
Yes, this dictionary is degenerate.  The solution $b$ of the basic variable $w_1$ is 0.
\medskip

\noindent\textbf{(C)}\\
The next dictionary looks like:
\[ \begin{array}{r| c c c c c c c}
w_1 & 0 &- 7x_1 &- 7x_5 &+ w_4 &- 2w_5 &- 2x_6 &+ 2w_3\\
w_2 & 2 & -x_1 & -x_5 & & & -x_6 & -w_3 \\
x_3 & 4 & -\frac{x_1}{2} & -\frac{w_5}{2} & -\frac{3w_4}{2} & -\frac{w_5}{2} & -x_6 & -w_3 \\
x_4 & 1 & -\frac{x_1}{2} & +x_5 & +\frac{w_4}{2} & +\frac{w_5}{2} & -x_6 & +2w_3 \\
x_2 & 2 & -\frac{x_1}{2} & -2x_5 & +\frac{w_4}{2} & -\frac{w_5}{2} && \\
\hline
z & 15 & -2x_1 & -3x_5 & -w_4 & -w_5 & +x_6 & +3w_3 \\
\end{array}\]
We can choose betwen $x_6$ and $w_3$ to be the entering variable.  We should choose $w_3$ to be the entering variable, because it increases the value of the objective function $z$ the most.  Since the leaving variable is automatically chosen based on the which basic variable constrains the entering variable the most, the value of the objective function does not depend on the choice of the leaving variable.

\medskip


\noindent\textbf{P2.}
\\ 
\noindent \textbf{(A)}
\\
$x_{N,j}$ is the entering variable and $x_{B,i}$ is the leaving variable.  This means, first, we need to solve the current equation in the dictionary given for $x_{B,i}$ for $x_{N,j}$.  We start with the equation:
$$
x_{B,i} = b_i + a_{i1}x_{N,1} + \cdots + a_{ij}x_{N,j} + \cdots + a_{in}x_{N,n}
$$
Solving for $x_{N,j}$ we get the equation:
$$
x_{N,j} = \frac{b_i}{-a_{ij}} + \frac{a_{i1}x_{N,1}}{-a_{ij}} + \cdots + \frac{a_{in}x_{N,n}}{-a_{ij}} + \frac{x_{B,i}}{-a_{ij}}
$$
The next step is to take that equation and plug it into any instances of $x_{N,j}$ on the nonbasic side of the dictionary.  If we use it in the equation for $x_{B,k}$ then we get the equation:
$$
x_{B,k} = b_k + a_{k1}x_{N,1} + \cdots + a_{kj}(\frac{b_i}{-a_{ij}} + \frac{a_{i1}x_{N,1}}{-a_{ij}} + \cdots + \frac{a_{in}x_{N,n}}{-a_{ij}} + \frac{x_{B,i}}{-a_{ij}}) + \cdots + a_{kn}x{N,n}
$$
Since we are trying to find the value of $x_{B,k}$, we set all of the nonbasic variables to 0 in the equation for $x_{B,k}$ yielding:
$$
x_{B,k} = b_k + a_{kj}(\frac{b_i}{-a_{ij}})
$$
\begin{tabular}{|c|c|c|}
\hline
Constant & Sign & Reason\\ \hline
$b_k$ & $\ge 0$ & Dictionary is feasible\\ \hline
$a_{kj}$ & Nothing may be said about its sign & Coefficient could be any value\\  \hline
$b_i$ & $\ge 0$ & Dictionary is feasible\\ \hline
$a_{ij}$ & $< 0$ & Entering variable must be constrained\\ \hline
\end{tabular}

\medskip

\noindent \textbf{(B)}
\\
We know $b_i \ge 0$ because the dictionary is feasible and $a_{ij} < 0$ because $x_{N,i}$ had to be constrained in order for $x_{N,j}$ to be chosen as the leaving variable.  We know then that the value of $x_{N,j} = (\frac{b_i}{-a_{ij}}) \ge 0$ because the dictionary is still feasible.  We also know $b_k \ge 0$ because the original dictionary was feasible.  What we don't know is the sign of $a_{kj}$.  It could be negative, positive, or the value could be 0.
\begin{enumerate}
\item If $a_{kj}$ is negative, then 
$
b_k - a_{kj} \left(\frac{b_i}{ - a_{ij}}\right)  \geq 0
$
which could violate the $\ge 0$ if $b_k < -a_{kj} \left(\frac{b_i}{ - a_{ij}}\right)$.  But since we did not choose $x_{B,k}$ as the leaving variable, this means 
$\frac{b_k}{-a_{kj}} \ge \frac{b_i}{-a_{ij}} \Rightarrow b_k \le -a_{kj}\frac{b_i}{-a_{ij}}$ and not strictly $ < -a_{kj}\frac{b_i}{-a_{ij}}$.

\item If $a_{kj}$ is 0 then 
$
b_k + a_{kj} \left(\frac{b_i}{ - a_{ij}}\right)  \geq 0 \Rightarrow
b_k \geq 0
$

\item If $a_{kj}$ is positive then
$
b_k + a_{kj} \left(\frac{b_i}{ - a_{ij}}\right)  \geq 0
$
which, based on the signs of the other constants, is true.
\end{enumerate}

\medskip


\noindent\textbf{(C)}
\\
If $x_{B,k}$ and $x_{B,i}$ are both possilbe leaving variables for $x_{N,j}$ then the value for $x_{N,j}$ in the equation of $x_{B,k}$ equals the value for $x_{N,j}$ in the equation of $x_{B,i}$.  This means:
$$
\frac{b_k}{-a_{kj}} = \frac{b_i}{-a_{ij}}
$$
If we choose $x_{B,i}$ to be the leaving variable instead of $x_{B,k}$ then the value of $x_{B,k}$ in the next dictionary is $b_k + a_{kj}(\frac{b_i}{ - a_{ij}})$.  Since $\frac{b_k}{-a_{kj}} = \frac{b_i}{-a_{ij}}$ we get:
$$
x_{B,k} = b_k + a_{kj}(\frac{b_i}{ - a_{ij}}) \Rightarrow x_{B,k} = b_k + a_{kj}(\frac{b_k}{-a_{kj}}) = b_k - b_k = 0 
$$
Because the value of $x_{B,k} = 0$, the next dictionary will be degenerate.

\bigskip

\noindent\textbf{P3}.

\noindent\textbf{(A)} A degenerate dictionary that is also unbounded. 
Recall that an unbounded dictionary does not have a leaving variable for some
choice of an entering variable.
\\
\[\begin{array}{r|cccc}
w_1 & 5 & +2x_2 & -3x_1\\ 
w_2 & 0 && +x_1 \\
\hline
z & 5 & +x2 & -x1
\end{array}\]

\medskip

\noindent\textbf{(B)} A degenerate dictionary $D$ which upon pivoting
yields another degenerate dictionary $D'$, but the objective value strictly
increases.
\\
\[\begin{array}{r|cccc}
w_1 & 5 & -2x_2 & -3x_1\\ 
w_2 & 0 && +x_1 \\
\hline
z & 5 & +x2 & -x1
\end{array}\]  

\medskip

\noindent\textbf{(C)} A non-degenerate dictionary $D$ which upon pivoting
yields another dictionary $D'$ but the value of the objective function
stays the same.
\\
\\
This is not possible.  The only way the value of the objective function $z'$ of a new dictionary $D'$ remains the same is if $b_i = 0$ in the equation $\frac{b_i}{a_{ij}}$.  This can only be the case if $D$ was degenerate.

\medskip

\noindent\textbf{(D)} A dictionary that is feasible but upon pivoting
yields an infeasible dictionary.
\\
\\
This is not possible.  Once you have a feasible dictionary you pivot to other feasible dictionaries.  You cannot pivot to an infeasible dictionary from a feasible dictionary.  You can, however, pivot to a degenerate dictionary from a feasible dictionary.

\medskip

\noindent\textbf{(E)} A dictionary that does not have leaving variable
(is unbounded) for one choice of entering variable but has a leaving
variable for a different choice of an entering variable.
\\
\[\begin{array}{r|cccc}
w_1 & 5 & +2x_2 & -3x_1\\ 
w_2 & 0 && +x_1 \\
\hline
z & 5 & +x2 & +x1
\end{array}\] 


\bigskip

\noindent\textbf{P4 (15 points)}  Consider the polyhedron below:
\[\begin{array}{ccccc}
 -x & + 2 y & + 2z & \leq & 2 \\
2 x & - y & + 2z & \leq & 2 \\
x & & & \geq & 0\\
& y& & \geq & 0 \\
& & z & \geq & 0 \\
\end{array}\]

\noindent\textbf{(A)}
\\
First we need to assign numbers to each constraint and saturate them to be equalities.
\[\begin{array}{ccccc}
 -x & + 2 y & + 2z & = & 2 \rightarrow 1\\
2 x & - y & + 2z & = & 2 \rightarrow 2\\
x & & & = & 0 \rightarrow 3\\
& y& & = & 0 \rightarrow 4 \\
& & z & = & 0 \rightarrow 5\\
\end{array}\]
We can then find the vertices by finding where each face--given by the constraint equations--intersects.  In this case we have three variables (3 dimensions) so we need to see where three equations intersect to find a vertex.  There will be ${5 \choose 3} = 10$ combinations of equations.\\
\\
If there are multiple dictionaries that describe a vertex, then that vertex is degenerate.  If there is only a single vertex to describe a vertex, then that vertex is non-degenerate.  This means if more than one combination of equations gives the same vertex, then that vertex is degenerate.  Otherwise, the vertex is not degenerate. 
\begin{center}
\begin{tabular}{|c|c|c|}
\hline
Equations Used & Vertex (x,y,z) & Degenerate or non-degenerate \\ \hline
3, 4, 5 & (0,0,0) & non-degenerate \\ \hline
1, 4, 5 & (-2,0,0) & non-degenerate \\ \hline
1, 3, 5 & (0,1,0) & non-degenerate \\ \hline
1, 3, 4 & (0,0,1) & degenerate \\ \hline
2, 4, 5 & (1,0,0) & non-degenerate\\ \hline
2, 3, 4 & (0,0,1) & degenerate \\ \hline
2, 3, 5 & (0,-2,0) & non-degenerate \\ \hline
1, 2, 3 & (0,0,1) & degenerate \\ \hline
1, 2, 4 & (0,0,1) & degenerate \\ \hline
1, 2, 5 & (2,2,0) & non-degenerate \\ \hline
\end{tabular}
\end{center}

\noindent\textbf{(B)} Consider the optimization problem:
\[ \begin{array}{rcccccc}
\max & 2x & + 3y & - 2z \\
\mathsf{s.t} &  -x & + 2 y & + 2z & \leq & 2 & \# Slack\ w_1\\
& 2 x & - y & + 2z & \leq & 2 & \# Slack\ w_2\\
& x & & & \geq & 0\\
& & y& & \geq & 0 \\
& & & z & \geq & 0 \\
\end{array}\]

Write down all the dictionaries corresponding to the degenerate
vertices. Use slack variables $w_1, w_2$ as indicated.
\\
The vertex that is degenerate is $(0,0,1)$. There are four dictionaries (four combinations of different equations) that produce that vertex.\\
\\
Using the equations 1, 3, and 4, we get the dictionary:
\[\begin{array}{r|ccccc}
w_2 & 0 & -3x & +3y & +w_1 \\
z & 1 & +\frac{x}{2} & -y & -\frac{w_1}{2} \\
\hline
\zeta & -2 & +x & +5y & +w_1 \\
\end{array}\]
Using the equations 2, 3, and 4, we get the dictionary:
\[\begin{array}{r|ccccc}
w_1 & 0 & +3x & -3y & +w_2 \\
z & 1 & -x & +\frac{y}{2} & -\frac{w_2}{2} \\
\hline
\zeta & -2 & +4x & +2y & +w_2
\end{array}\]
Using the equations 1, 2, and 3, we get the dictionary:
\[\begin{array}{r|ccccc}
y & 0 & +x & -\frac{w_1}{3} & +\frac{w_2}{3} \\
z & 1 & -\frac{x}{2} & -\frac{w_1}{6} & -\frac{w_2}{3} \\
\hline
\zeta & -2 & +6x & -\frac{2w_1}{3} & +\frac{5w_2}{3}
\end{array}\]
Using the equations 1, 2, and 4, we get the dictionary:
\[\begin{array}{r|ccccc}
x & 0 & +y & +\frac{w_1}{3} & -\frac{w_2}{3} \\
z & 1 & -\frac{y}{x} & -\frac{w_1}{3} & -\frac{w_2}{2} \\
\hline
\zeta & -2 & +6y & +\frac{4w_1}{3} & -\frac{w_2}{3}
\end{array}\]
\noindent\textbf{(C)} Draw a graph whose nodes are the vertices
described in (A) with edges between adjacent vertices. 
\\
You can only use the vertices that comply with the constraints which means they are in the feasible region.  This leaves the following vertices in the feasible region: \{(0,0,0), (0,1,0), (0,0,1), (1,0,0), (2,2,0)\}.


\medskip

\noindent\textbf{(D, extra credit)} Given a polyhedron $P$, and for
each vertex of the polyhedron,  can you  write down an
objective function that is uniquely maximized only at that vertex and
no other vertex of $P$? 


\medskip

\noindent\textbf{(E, extra credit)} For any polyhedron $P$, the
polyhedral graph (also called its skeleton) is one where the nodes
form the vertices of the polyhedron, and the edges connect adjacent
vertices. Prove that this graph is strongly connected for any
$P$. I.e, given any two vertices $\mathbf{v}_1$ and $\mathbf{v_2}$
there is a path between them in this graph.

\medskip

\noindent\textbf{(F, extra credit)}  Prove that the graph in part (E)
for a d-dimensional polyhedron $P$ has the property that if any subset
of $d-1$ or fewer vertices in the graph are removed, it will still
remain strongly connected (This is called Balinski's theorem).

To illustrate this, draw the skeleton of a cube and remove any two
vertices from this skeleton. You will notice that there is a path in
this graph between any pair of remaining vertices.



\end{document}
