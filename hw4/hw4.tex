\documentclass[11pt]{article}
\usepackage{fullpage}
\usepackage[margin=0.5in]{geometry}
\usepackage{amsmath,amssymb,amsfonts,graphicx,subfig,float,listings}
\renewcommand\vec[1]{\mathbf{#1}}
\usepackage{listings}
\usepackage[table]{xcolor}
\usepackage{framed,comment}
\newcommand\vx {\mathbf{x}}
\newcommand\vy {\mathbf{y}}
\newcommand\vb{\mathbf{b}}
\newcommand\vc{\mathbf{c}}
\newcommand\red[1]{\textcolor{red}{#1}}
\specialcomment{solution}{\bigskip\begin{leftbar}\par\noindent\textbf{Solution.} }{\end{leftbar} }
\excludecomment{solution}

\usepackage{color} %red, green, blue, yellow, cyan, magenta, black, white
\definecolor{mygreen}{RGB}{28,172,0} % color values Red, Green, Blue
\definecolor{mylilas}{RGB}{170,55,241}
\begin{document}

\lstset{language=Matlab,%
    %basicstyle=\color{red},
    breaklines=true,%
    morekeywords={matlab2tikz},
    keywordstyle=\color{blue},%
    morekeywords=[2]{1}, keywordstyle=[2]{\color{black}},
    identifierstyle=\color{black},%
    stringstyle=\color{mylilas},
    commentstyle=\color{mygreen},%
    showstringspaces=false,%without this there will be a symbol in the places where there is a space
    numbers=left,%
    numberstyle={\tiny \color{black}},% size of the numbers
    numbersep=9pt, % this defines how far the numbers are from the text
    emph=[1]{for,end,break},emphstyle=[1]\color{red}, %some words to emphasise
    %emph=[2]{word1,word2}, emphstyle=[2]{style},    
}

\begin{tabular}{l}
	\textbf{CSCI 5654-Fall16}: Assignment \# 4 \\
	\textbf{Your Name: Robert Werthman} \phantom{Supercalifragilisticexpialidocius Smith} \\
	\hline \\[10pt]
\end{tabular}

\noindent\textbf{P1.}
\bigskip

\noindent\textbf{(A)}
\\
Let $\max( 2 x_1 + 3 x_2 - 5 x_3,\ x_1,\ x_2,\ 2) \leq t$.  We can then form the following linear program:
\[\begin{array}{rllllll}
\min & t \\
\mathsf{s.t.} 
& +2x_1 & +3x_2 & -5x_3 & \leq t \\
& +2x_1 & -x_2 & +x_3 & \leq t \\
& x_1, & x_2 & & \leq t \\
\end{array}\]

\medskip 

\noindent\textbf{(B)}
\\
Let $t_1$, $t_2$, $t_3$, $t_4 \geq 0$. \\
Let $|x_1 + x_2| \leq t_1$, $| x_2 - x_3 | \leq t_2$, $| x_3 - x_1 | \leq t_3$, $| x_1 + x_2 + x_3 | \leq t_4$.\\
You now have the linear problem
\[\begin{array}{rllllll}
\min & +t_1 & +t_2 & +t_3 & +t_4 \\
\mathsf{s.t.} 
& +x_1 & +x_2 & & & \leq t_1 \\
& -x_1 & -x_2 & & & \leq t_1 \\
& & +x_2 & -x_3 & & \leq t_2 \\
& & -x_2 & +x_3 & & \leq t_2 \\
& -x_1 & & +x_3 & & \leq t_3 \\
& +x_1 & & -x_3 & & \leq t_3 \\
& +x_1 & +x_2 & +x_3 & & \leq t_3 \\
& -x_1 & -x_2 & -x_3 & & \leq t_3 \\
& t_1, & t_2 & t_3 & t_4 & \geq 0 \\
\end{array}\]

\medskip

\noindent\textbf{(C)}
\\
Let $\max( | x_1 | ,\ |x_2| ,\ |x_3|,\ |x_1+
  x_2 |) \leq t$. \\
You now have the linear problem
\[\begin{array}{rllllll}
\min & t \\
\mathsf{s.t.} 
& +x_1 & -x_2 & & \leq 5 \\
& & +x_2 & & \leq 3 \\
& +x_1 & & & \leq t \\
& -x_1 & & & \leq t \\
& & +x_2 & & \leq t \\
& & -x_2 & & \leq t \\
& & & +x_3 & \leq t \\
& & & -x_3 & \leq t \\
& +x_1 & +x_2 & & \leq t \\
& -x_1 & -x_2 & & \leq t \\
& t & & & \geq 0 \\
\end{array}\]


\bigskip
\newpage
\noindent\textbf{P2.}
\\
\lstinputlisting[basicstyle=\scriptsize]{p2.m}
\begin{center}
\begin{tabular}{|c|c|}
\hline
Coefficient & Value \\ \hline
$a_0$ & 8.704148513061227e-14 \\ \hline
$a_1$ & -3.375077994860476e-14 \\\hline
$a_2$ & -1.474376176702208e-13 \\\hline
$a_3$ & 2.131628207280301e-13 \\\hline
$a_4$ & -1.243449787580175e-13 \\\hline
$a_5$ & -1.056932319443149e-13 \\\hline
$a_6$ & 1.376676550535194e-13 \\\hline
$a_7$ & -3.979039320256561e-13 \\\hline
$a_8$ & 3.179678742526448e-13 \\\hline
$a_9$ & 6.430411758628907e-13 \\\hline
$a_{10}$ & 0.999999999999309 \\\hline
$b_0$ & 1.989519660128281e-13 \\\hline
$b_1$ & -2.060573933704291e-13 \\\hline
$b_2$ & 1.847411112976261e-13 \\\hline
$b_3$ & -8.526512829121202e-14 \\\hline
$b_4$ & -1.207922650792170e-13 \\\hline
$b_5$ & 7.815970093361102e-14 \\
\hline
\end{tabular}
\end{center}
\begin{figure}[H]
\centering
  \subfloat[Linear Regression Predictions]{
  \includegraphics[scale=.7]{predictions.png}
  }
\end{figure}
\begin{figure}[H]
\centering
  \subfloat[Residuals]{
  \includegraphics[scale=.8]{residuals.png}
  }
\end{figure}

\newpage

\noindent\textbf{P3.}\\

\noindent\textbf{(A)}

\begin{enumerate}
\item 
  \[\begin{array}{r|ccccccccccccc}
  x_1 & +2 &  -x_4 & +x_5 & -x_6 & & & +w_{6}\\
  x_2 & -8 &  & +2x_5 & +x_6 & +w_{4} & +w_{5} & +w_{6} \\
  x_3 & +4 &  +x_4 & -2x_5 & & & -w_5 & -w_6 \\
  w_1 & -7 &  +x_4 & +x_5 & +3x_6 & +w_4 & +w_5 & & \\
  w_2 & -3 &  +2x_4 & & +x_6 & & & -w_6 \\
  w_3 & 2 &  +x_4 & -2x_5 & +x_6 & & -w_5 & -w_6 \\ 
  \hline
  \zeta & 16 & & -6x_5 & & -3w_4 & -2w_5 & -5w_6 \\
  \end{array}\]
\item 
  \[\begin{array}{r|ccccccccccccc}
  x_1 & -1 & -x_3 & +x_4 & +2x_6 & -w_1 & & +w_4 \\ 
  x_2 & -4 & -x_3 & +x_4 & +x_6 & & & +w_4 \\
  x_5 & 0 & -x_3 & & +2x_6 & -w_1 & +w_2 & +w_4 \\
  w_3 & -2 & +x_3 & & +x_6 & & & & \\
  w_5 & 7 & +x_3 & -x_4 & -5x_6 & +2w_1 & -w_2 & -2w_4 \\
  w_6 & -3 & & 2x_4 & +x_6 & & -w_2 & & \\
  \hline
  \zeta & 14 & +4x_3 & -6x_4 & -6x_6 & +2w_1 & & -5w_4 \\
  \end{array}\]
\item 
  \[\begin{array}{r|ccccccccccccc}
  x_1 & -1 & & +x_4 & +x_5 & & -w_2 & & \\
  x_2 & -6 & +x_3 & +x_4 & +x_5 & +w_1 & -w_2 & -w_3 \\
  x_6 & 2 & -x_3 & & & & & +w_3 \\
  w_4 & -4 & +3x_3 & & +x_5 & +w_1 & -w_2 & -2w_3 \\
  w_5 & 5 & & -x_4 & -2x_5 & & +w_2 & -w_3 \\
  w_6 & -1 & -x_3 & +2x_4 & & & -w_2 & +w_3 \\
  \hline
  \zeta & 22 & -5x_3 & -6x_4 & -5x_5 & -3w_1 & +5w_2 & +4w_3 \\
  \end{array}\]
\end{enumerate}


\noindent\textbf{(B)}

\begin{enumerate}
\item
  \[\begin{array}{r|ccccccccccccc}
  x_3 & 2 & & & -x_6 & +w_3 & & & \\
  x_4 & 6 & & +x_2 & -2x_6 & +w_3 & -w_4 & \\
  x_5 & 4 & -x_1 & & & -w_3 & & -w_5 \\
  w_1 & 3 & -x_1 & +x_2 & +x_6 & & & \\
  w_2 & 9 & -2x_1 & +x_2 & -2x_6 & & -w_4 & -w_5 \\
  w_6 & 0 & +2x_1 & +x_2 & -x_6 & +2w_3 & -w_4 & +w_5 \\
  \hline
  \zeta & -8 & -2x_1 & -4x_2 & +4x_6 & -2w_3 & +w_4 & \\
  \end{array}\]
\item
$x_6$ will be the entering variable because it has the largest coefficient.
\item
  The equation for the entering variable column $a_i$ is $a_i = -B^{-1}Ne_i$.

\item
The leaving variable with be $w_6$ because it constrains the entering variable the most.
\item
The basic variables will be $x_3,x_4,x_5,w_1,w_2,x_6$.
\end{enumerate}

\noindent\textbf{(C)}
\\
\[\begin{array}{r|ccccccccccccc}
x_5 & 1 & +x_1 & & & -x_4 & +w_2 & \\
x_6 & 4 & & +x_2 & +x_3 & -x_4 & & -w_4 \\
w_1 & 7 & -x_1 & +2x_2 & +x_3 & -x_4 & & -w_4 \\
w_3 & 2 & & +x_2 & +2x_3 & -x_4 & & -w_4 \\
w_5 & 1 & -2x_1 & -x_2 & -2x_3 & +2x_4 & -w_2 & +w_4 \\
w_6 & 1 & & +x_2 & +x_3 & +x_4 & -w_2 & -w_4 \\
\hline
\zeta & 4 & -2x_1 & -2x_2 & & -2x_4 & & -w_4 \\
\end{array}\]
\bigskip

\noindent\textbf{(D)}
\\
This produces a new objective function: $(-2+u)x_1 + (-3+u)x_2 -x_3 -x_4 + (1+u)x_6$.  To make sure the dictionary is final and feasible we have to ensure that the objective row coefficients are $\leq 0$ and the constant column values are $\geq 0$.  Trying different values of $\mu$ I found a bound of $-1 \leq \mu \leq 0$ as values of the $\mu$ that kept the dictionary final and feasible.
\bigskip

\noindent\textbf{(E)} 
\\
We need to ensure that the constant column of the dictionary is $\geq 0$ to ensure the dictionary is feasible.  Trying different values of $\lambda$ I found a boud of $-1 \leq \lambda \leq +1$.
\newpage
\lstinputlisting[basicstyle=\scriptsize]{p3.m}


\end{document}
