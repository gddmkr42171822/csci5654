\documentclass[12pt]{article}

\usepackage{amsmath,listings}
\usepackage[margin=0.5in]{geometry}

\begin{document}

\noindent
Robert Werthman\\
CSCI 5654\\
Homework 1\\

\section*{P1.}
\begin{enumerate}
  \item First change the problem into a maximization problem:
        $$
          \text{minimize } 3x_{1} - 5x_{2} \Rightarrow 
          \text{maximize } -3x_{1} + 5x_{2}
        $$
  \item Change the constraints to $\leq$:
        $$4x_{1} + x_{2} \geq -4 \Rightarrow -4x_{1} - x_{2} \leq 4$$
        $$2x_{1} - x_{2} \geq -8 \Rightarrow -2x_{1} + x_{2} \leq 8$$
  \item Make sure the variables have non-negativity constraints:
        $$x_{2} \mapsto x^{+}_{2} - x^{-}_{2}$$
        $$x^{+}_{2}, x^{-}_{2} \geq 0$$
        Replace $x_{2}$ with $x^{+}_{2} - x^{-}_{2}$ in the problem.
\end{enumerate}
The problem can now be written in standard form as:
\begin{align*}
  \textbf{maximize } -3x_{1} + 5x^{+}_{2} - 5x^{-}_{2} &\\
  \textbf{s.t. }     -4x_{1} - x^{+}_{2} + x^{-}_{2} &\leq 4\\
                     -2x_{1} + x^{+}_{2} - x^{-}_{2} &\leq 8\\
                     x_{1} + 2x^{+}_{2} - 2x^{-}_{2} &\leq 4\\
                     x_{1}, x^{+}_{2}, x^{-}_{2} &\geq 0
\end{align*}
\newpage

\section*{P2.}
\subsection*{Slack, initial dictionary, feasibility}
I will add the slack variables $w_{1}$, $w_{2}$, and $w_{3}$ to the constraints and let $z$ be the value of the objective function.\\
Rewriting the problem with the slack variables gives:
\begin{align*}
  \textbf{maximize } z &= -3x_{1} + 5x^{+}_{2} - 5x^{-}_{2}\\
  \textbf{s.t. }     w_{1} &= 4 + 4x_{1} + x^{+}_{2} - x^{-}_{2}\\
                     w_{2} &= 8 + 2x_{1} - x^{+}_{2} + x^{-}_{2}\\
                     w_{3} &= 4 - x_{1} - 2x^{+}_{2} + 2x^{-}_{2}\\
                     &x_{1}, x^{+}_{2}, x^{-}_{2}, w_{1}, w_{2}, w_{3} \geq 0
\end{align*}
The first dictionary of this problem can be written as:\\
\\
\begin{tabular}{l}
  $w_{1} = 4 + 4x_{1} + x^{+}_{2} - x^{-}_{2}$\\
  $w_{2} = 8 + 2x_{1} - x^{+}_{2} + x^{-}_{2}$\\
  $w_{3} = 4 - x_{1} - 2x^{+}_{2} + 2x^{-}_{2}$\\
  \hline
  $z = 0 - 3x_{1} + 5x^{+}_{2} - 5x^{-}_{2}$\\
\end{tabular}
\\
\\
A solution to this dictionary is feasible if and only if all the right-hand sides are nonnegative as set by the non-negativity constraints.\\
\\
If we set the non-basic variables $x_{1}, x^{+}_{2}, x^{-}_{2}$ to 0 then we have the solutions: $w_{1} = 4$, $w_{2} = 8$, $w_{3} = 4$, and $z = 0$.\\
\\
Since the non-negativity constraints are respected, this dictionary is feasible.
\subsection*{Initial pivoting}
To pivot the dictionary we need to choose a variable from the objective funtion that if increased the value $z$ of the objective function increases.  This is the case because we are trying to maximize the value of the objective function.  In this case, we will choose $x^{+}_{2}$ because it is the only variable that will increase the value of $z$.  The other variables all of $-$ in front which means that if we increased those variables, we would decrease the value $z$ of the objective function.\\
\\
To find the leaving variable we need to solve for $x^{+}_{2}$ in all of the equations of the dictionary where their is a constraint on how much $x^{+}_{2}$can increase.  We then choose the variable where $x^{+}_{2}$ has the lowest upper limit.  We set all of the other variables to 0.
\begin{align*}
  w_{1} &= 4 + 4x_{1} + x^{+}_{2} - x^{-}_{2} \Rightarrow \text{No constraint on the increase of } x^{+}_{2} \\
  w_{2} &= 8 + 2x_{1} - x^{+}_{2} + x^{-}_{2} \Rightarrow x^{+}_{2} \leq 8\\
  w_{3} &= 4 - x_{1} - 2x^{+}_{2} + 2x^{-}_{2} \Rightarrow x^{+}_{2} \leq 2\\
\end{align*}
In this case, we choose $w_{3}$ as the leaving variable because it gives $x^{+}_{2}$ the lowest upper bound constraint.\\
\\
Once we have the entering and leaving variables, we need to find the new dictionary after pivoting.  To do this we first solve the equation of the leaving variable for the entering variable:
$$
  w_{3} = 4 - x_{1} - 2x^{+}_{2} + 2x^{-}_{2} \Rightarrow 
  x^{+}_{2} = 2 - \frac{x_{1}}{2} + x^{-}_{2} - \frac{w_{3}}{2}
$$
We then substitute that equation in for all places where the entering variable occurrs in the dictionary:
\begin{align*}
  w_{1} &= 4 + 4x_{1} + (2 - \frac{x_{1}}{2} + x^{-}_{2} - \frac{w_{3}}{2}) - x^{-}_{2}\\
  w_{2} &= 8 + 2x_{1} - (2 - \frac{x_{1}}{2} + x^{-}_{2} - \frac{w_{3}}{2}) + x^{-}_{2}\\
  x^{+}_{2} &= 2 - \frac{x_{1}}{2} + x^{-}_{2} - \frac{w_{3}}{2}\\
  z &= 0 - 3x_{1} + 5(2 - \frac{x_{1}}{2} + x^{-}_{2} - \frac{w_{3}}{2}) - 5x^{-}_{2}
\end{align*}
After some algebra, the new dictionary after pivoting looks like:\\
\\
\begin{tabular}{l}
  $w_{1} = 6 + \frac{7}{2}x_{1} - \frac{w_{3}}{2}$\\[.5em]
  $w_{2} = 6 + \frac{3}{2}x_{1} + \frac{w_{3}}{2}$\\[.5em]
  $x^{+}_{2} = 2 - \frac{x_{1}}{2} + x^{-}_{2} - \frac{w_{3}}{2}$\\[.5em]
  \hline
  $z = 10 - \frac{11}{2}x_{1} - \frac{5}{2}w_{3}$\\
\end{tabular}
\newpage

\section*{P3.}
\subsection*{Linear programming model of the problem}
The client needs to invest money in each of investments options in order to maximize their profit.  Let $x_{i}$ be the amount of money invested in the investment option with ID $i$.  Let $epu_{i}$ be the expected profit per unit for the investment option with ID $i$.  Let $pu_{i}$ be the price per unit for the investment option with ID $i$.  The objective function for this maximization problem can then be written as:
$$
  \textbf{maximize } \sum_{i = 1}^{15} \frac{epu_{i}}{pu_{i}}x_{i}
$$
This says that the total expected profit of all the investments is equal to the sum of the amount of money invested in each investment option divided by the price per unit for each investment option multiplied by the expected profit per unit for each investment option.  This is subject to the constraints:
\begin{enumerate}
\itemsep0em
  \item $\sum_{i = 1}^{15} x_{i} \leq 10000$\\
  - Total investment is at most \$10,000.
  \item $1500 \leq x_{1} + x_{4} + x_{10} + x_{13} \leq 3500$\\
  - The minimum and maximum investment in risk category A.
  \item $4500 \leq x_{2} + x_{5} + x_{8} + x_{9} + x_{14} \leq 6500$\\
  - The minimum and maximum investement in risk category B.
  \item $1000 \leq x_{3} + x_{7} + x_{11} + x_{15} \leq 3000$
  - The minimum and maximum investement in risk category C.
  \item $500 \leq x_{6} + x_{12} \leq 2500$\\
  - The minimum and maximum investement in risk category D.
  \item $0 \leq x_{1} + x_{8} + x_{11} \leq 3000$\\
  - The minimum and maximum investement in investement market Tech.
  \item $0 \leq x_{2} + x_{3} + x_{5} + x_{6} + x_{7} + x_{15} \leq 4000$\\
  - The minimum and maximum investement in investement market Finance.
  \item $0 \leq x_{4} + x_{9} + x_{13} \leq 5000$\\
  - The minimum and maximum investement in investement market PetroChem.
  \item $0 \leq x_{10} + x_{12} + x_{14} \leq 7000$\\
  - The minimum and maximum investement in investement market Automobile.
  \item $2000 \leq x_{1} + x_{2} + x_{3} + x_{6} + x_{7} + x_{8} + x_{9} + x_{10} + x_{11} + x_{13} \leq 10000$\\
  - The minimum and maximum investement in EcoFriendly.
  \item $x_{1},...,x_{15} \geq 0$\\
  - You cannot invest a negative amount of money in an investment option.
\end{enumerate}

\subsection*{}

\subsection*{Solution}
I used the PyGLPK python library to solve this problem.  The problem turned out to be feasible with none of the constraints contradicting each other.  The problem was also unbounded because the maximum of the objective functions was not $\infty$ or $-\infty$.\\ 
\begin{itemize}
  \item The optimal value $z$ of the objective function to maximize profit is \$8513.47.
  \item \$3000 dollars should be invested in the investment option with ID 3 ($x_{3}$).
  \item \$1500 dollars should be invested in the investment option with ID 10 ($x_{10}$).
  \item \$500 dollars should be invested in the investment option with ID 12 ($x_{12}$).
  \item \$5000 dollars should be invested in the investment option with ID 14 ($x_{14}$).
  \item Nothing should be invested in the other investment options.
\end{itemize}
\newpage
\section*{P4.}
\subsection*{(A)}
Let two solutions $y, z \in F$ and let the solution $x = \lambda y + (1 - \lambda)z$ which means $x \in F$.  Then we can say
$$
Ax = A(\lambda y + (1 - \lambda)z) = \lambda Ay + (1 - \lambda)Az \leq \lambda b + (1 - \lambda)b = b\\
$$
The set of feasible solutions $F$ is convex, because any solution $x$ made up of two feasible solutions $y, z \in F$ and of the form $\lambda y + (1 - \lambda)z$ respects the equation $Ax \leq b$.  This makes $x$ a feasible solution $\in F$ and the set $F$ convex.

\subsection*{(B)}
Both optimal solutions are in a convex set of $O : \{x | c^tx = z^*\}$.  If there are infinitely many optimal solutions in $O$ then there exists a point equal to $\lambda x_{1} + (1 - \lambda)x_{2}$ along the line connecting $x_{1}$ and $x_{2}$ that is also equal $z^*$.\\
\\
\noindent
Let $y = x_{1}$ and $w = x_{2}$ then the optimal objective value is
$$
  z^* = C^ty = C^tw
$$
Let $x = \lambda y + (1 - \lambda)w$ then we can say 
$$ 
  C^tx = C^t(\lambda y + (1 - \lambda)w)) = \lambda C^ty + (1 - \lambda)C^tw = \lambda z^* (1 - \lambda) z^* = z^*
$$
It was shown that the solution $x$ of the form $\lambda y + (1 - \lambda)w$ equals the optimal objective value $z^*$.  This means $x$ is in the set $O$ and that the set $O$ is convex.  This means there are infinitely many optimal solutions of the form $\lambda y + (1 - \lambda)w$ where $\lambda \in [0,1]$ in the optimal solution set $O$.

\newpage

\section*{P5.}
\subsection*{(A)}
\subsection*{Linear programming model of the problem}
The goal of this problem is to minimize the total cost of food while meeting all of the caloric nutrient needs.  Let $x_{i}$ represent the food item $i$.  Let $ps_{i}$ represent the price per serving of food item $i$.  The objective function of this minimization problem can then be written as:
$$
  \textbf{minimize } .5x_{0} + 2.5x_{1} + .25x_{2} + .2x_{3} + .6x_{4}
$$
This says that to minimize the cost of the food that is purchased we need to choose the optimal quantities of each food item.  This is subject to the constraints:
\begin{enumerate}
\itemsep0em
  \item $1800 \leq 300x_{0} + 550x_{1} + 450x_{2} + 25x_{3} + 300x_{4} \leq 2200$\\
  - The minimum and maximum calories consumed for all of the food items.
  \item $50 \leq 20x_{0} + 25x_{1} + 25x_{2} + 4x_{3} + 15x_{4} \leq 100$\\
  - The minimum and maximum carbs consumed for all of the food items.
  \item $30 \leq 5x_{0} + 8x_{1} + 4x_{2} + 2x_{3} + 3x_{4} \leq 80$\\
  - The minimum and maximum protein consumed for all of the food items.
  \item $60 \leq 10x_{0} + 20x_{1} + 5x_{2} + .5x_{3} + .5x_{4} \leq 100$\\
  - The minimum and maximum fats consumed for all of the food items.
  \item $3 \leq .1x_{0} + .9x_{1} + .1x_{2} + .1x_{3} + .1x_{4} \leq 5$\\
  - The minimum and maximum sodium consumed for all of the food items.
  \item $x_{0}, x_{1}, x_{2}, x_{3}, x_{4} \geq 0$\\
  - The amount of each nutrient consumed cannot be negative.
\end{enumerate}

\subsection*{Solution}
I used the PyGLPK python library to solve this problem.  It used the Simplex algorithm to solve Linear Programming problems.  It gave the following answers:
\begin{itemize}
  \item The optimal value $z$ of the objective function to minimize cost is \$7.96.
  \item 2.76 servings of Ramen $x_{1}$ should be purchased.
  \item .49 servings of Rice $x_{2}$ should be purchased.
  \item 4.67 servings of Brocolli $x_{3}$ should be purchased.
  \item No servings of cookies $x_{0}$ or cornflakes $x_{4}$ should be purchased.
\end{itemize}

\subsection*{(B)}
The constraints below should be added to ensure that no single food accounts for more than 50\% of the total caloric intake:
\begin{enumerate}
\itemsep0em
  \item $300x_{0} \leq 0.5(300x_{0} + 550x_{1} + 450x_{2} + 25x_{3} + 300x_{4})$\\
  - The calories from Cookies should not account for more than 50\% of the total caloric intake. 
  \item $550x_{1} \leq 0.5(300x_{0} + 550x_{1} + 450x_{2} + 25x_{3} + 300x_{4})$\\
  - The calories from Ramen should not account for more than 50\% of the total caloric intake. 
  \item $450x_{2} \leq 0.5(300x_{0} + 550x_{1} + 450x_{2} + 25x_{3} + 300x_{4})$\\
  - The calories from Rice should not account for more than 50\% of the total caloric intake. 
  \item $25x_{3} \leq 0.5(300x_{0} + 550x_{1} + 450x_{2} + 25x_{3} + 300x_{4})$\\
  - The calories from Broccoli should not account for more than 50\% of the total caloric intake. 
  \item $300x_{4} \leq 0.5(300x_{0} + 550x_{1} + 450x_{2} + 25x_{3} + 300x_{4})$\\
  - The calories from CornFlakes should not account for more than 50\% of the total caloric intake.
\end{enumerate}

\subsection*{Solution}
I used the command line tool \textbf{glpsol} that comes with GLPK to solve the problem.  I used the GNU MathProg modeling language (GMPL) to create a .mod file with the linear programming model of the problem.  The solver said that with these new constraints there was no feasible solution to this problem.

\subsection*{(C)}
This will be a proof by contradiction.  Let a solution vector $x = (x_0, x_1, x_2, x_3, x_4)$ and two feasible solution vectors $y = (2, 3, 2, 0, 0)$ and $z = (2, 3, 0, 0, 2)$.  If the set of $x, y$ is convex then there is a $w = \lambda x + (1 - \lambda )y$.  Let $\lambda = .5$, the midpoint of a line between $y$ and $x$.  Then
$$
  w = .5(x + y) = (\frac{2+2}{2}, \frac{3+3}{2}, \frac{2+0}{2}, \frac{0+0}{2}, \frac{2+0}{2}) = (2, 3, 1, 0, 1)
$$
Only two variables in $w$ are $> 1$ which violates the constraint of at least 3 variables (foods) have a serving size $> 1$.  This means the set of solutions with more than 1 serving of at least 3 foods is not convex and therefore cannot be a constraint of a linear program.

\newpage

\lstinputlisting[basicstyle=\small]{p3.py}

\lstinputlisting[basicstyle=\small]{p5.py}

\lstinputlisting[basicstyle=\small]{p5.mod}

\lstinputlisting[basicstyle=\small]{p5.sol}

\end{document}