\documentclass[11pt]{article}
\usepackage{fullpage}
\usepackage{graphicx,enumerate}
\usepackage{amsmath,amssymb,amsfonts}
\renewcommand\vec[1]{\mathbf{#1}}
\usepackage{listings}
\usepackage[table]{xcolor}
\usepackage{framed,comment}
\newcommand\vx {\mathbf{x}}
\newcommand\vy {\mathbf{y}}
\newcommand\vb{\mathbf{b}}
\newcommand\vc{\mathbf{c}}
\newcommand\red[1]{\textcolor{red}{#1}}
\specialcomment{solution}{\bigskip\begin{leftbar}\par\noindent\textbf{Solution.} }{\end{leftbar} }
\excludecomment{solution}

\usepackage{color} %red, green, blue, yellow, cyan, magenta, black, white
\definecolor{mygreen}{RGB}{28,172,0} % color values Red, Green, Blue
\definecolor{mylilas}{RGB}{170,55,241}

\begin{document}

\lstset{language=Matlab,%
    breaklines=true,%
    morekeywords={matlab2tikz},
    keywordstyle=\color{blue},%
    morekeywords=[2]{1}, keywordstyle=[2]{\color{black}},
    identifierstyle=\color{black},%
    stringstyle=\color{mylilas},
    commentstyle=\color{mygreen},%
    showstringspaces=false,%without this there will be a symbol in the places where there is a space
    numbers=left,%
    numberstyle={\tiny \color{black}},% size of the numbers
    numbersep=9pt, % this defines how far the numbers are from the text
    emph=[1]{for,end,break},emphstyle=[1]\color{red}, %some words to emphasise
}

\begin{tabular}{l}
	\textbf{CSCI 5654-Fall16}: Assignment \#5 \\
	\textbf{Robert Werthman} \phantom{Supercalifragilisticexpialidocius Smith} \\
	\hline \\[10pt]
\end{tabular}

\noindent\textbf{P1.}
\\
\noindent\textbf{(A)}
\\
Solve the initial linear equation and then branch on any variables that have a fractional value.  Add the constrains of the fractional variables, solve the linear equation, and repeat until the solution is integral and not fractional.\\ \\
\begin{tabular}{|l|l|l|}
\hline
Branches & Solution & Optimal value \\ \hline
& $x = [-5\quad-5\quad1.5\quad-.5]$ & 2.5 \\ \hline
$x_4 \leq -1$ & $x = [-5\quad-5\quad2\quad-1]$ & 2\\ \hline
$x_4 \geq 0$ & $x = [-2\quad-4\quad0\quad0]$ & 2 \\ \hline
\end{tabular}
\\ \\
The optimal value of the objective function is 2.  One solution that leads to this value is: 
$$
x_1 = 1 \quad x_2 = 1 \quad x_3 = 1 \quad x_4 = 1
$$


\noindent\textbf{(B)}
\\
\begin{tabular}{|l|l|l|}
\hline
Branches & Solution & Optimal value \\ \hline
& $x = [1.3333\quad1\quad1\quad.6667]$ & 6.3333 \\ \hline
$x_4 \geq 1$ & $x = [1\quad1\quad1\quad1]$ & 6 \\ \hline
$x_4 \leq 0$ & $x = [0\quad1\quad1\quad0]$ & 3 \\ \hline
$x_1 \leq 1$ & $x = [1\quad1\quad1\quad1]$ & 6 \\ \hline
$x_1 \geq 2$ & Infeasible & \\ \hline
\end{tabular}
\\ \\
The optimal value of the objective function is 6.  One solution that leads to this value is:
$$
x_1 = 1 \quad x_2 = 1 \quad x_3 = 1 \quad x_4 = 1
$$

\newpage
\noindent\textbf{P2.}
\\
\noindent\textbf{Dictionary \# 1}
\\
First, we choose $x_1$ because it is a variable with a fractional solution.  We rewrite the equation for $x_1$ as:
$$
0.666667 x_{5} - 0.333333 x_{4} + x_1 = 0.666666666667
$$
Next we rewrite the above equation in terms of an integer part and a fractional part.
$$
(0x_5 - x_4 + x_1) + (0.666667 x_{5} + 0.777777 x_{4}) = 0 + 0.666666666667
$$
The fractional part $(0.666667 x_{5} + 0.777777 x_{4}) \geq 0.666666666667$.
The cutting plane is then given by the equation:
$$
(0.666667 x_{5} + 0.777777 x_{4}) + w_6 = 0.666666666667
$$

\noindent\textbf{Dictionary \# 2} 
\\
Equations for variables with fractional solutions:
\begin{align*}
-0.333333 x_{8} - 0.666667 x_{9} + 0.333333 x_{3} + x_{4} = 4.33333333333 \\
0.333333 x_{8} - 0.333333 x_{9} + 2.666667 x_{3} + x_{5} = 8.66666666667 \\
0.333333 x_{8} + 0.666667 x_{9} - 0.333333 x_{3} + x_{1} = 5.66666666667 \\
-0.333333 x_{8} + 0.333333 x_{9} - 2.666667 x_{3} + x_{2} = 1.33333333333 \\
\end{align*}
Equations written with integral and fractional parts:
\begin{align*}
(-x_8 - x_9 +0x_3 + x_4) + (0.777777 x_{8} + 0.444443 x_9 + 0.333333 x_{3}) = 4 + .33333333333 \\
(0x_8 - x_9 + 2x_3 +x_5) + (0.333333 x_{8} + 0.777777 x_{9} + 666667 x_{3}) = 8 + .66666666667 \\
(0x_8 + 0x_9 - x_3 + x_1) + (0.333333 x_{8} + 0.666667 x_{9} + 0.777777 x_{3}) = 5 + .66666666667 \\
(-x_8 + 0x_9 -3x_3 + x_2) + (0.777777 x_{8} + 0.333333 x_{9} + 0.444443 x_{3}) = 1 + .33333333333 \\
\end{align*}
Cutting planes for the above equations:
\begin{align*}
(0.777777 x_{8} + 0.444443 x_9 + 0.333333 x_{3}) + w_{10}= .33333333333 \\
(0.333333 x_{8} + 0.777777 x_{9} + 666667 x_{3}) + w_{11}= .66666666667 \\
(0.333333 x_{8} + 0.666667 x_{9} + 0.777777 x_{3}) + w_{12}= .66666666667 \\
(0.777777 x_{8} + 0.333333 x_{9} + 0.444443 x_{3}) + w_{13}= .33333333333 \\
\end{align*}

\medskip

\newpage
\noindent\textbf{P3.}
\\
\noindent\textbf{(A)}
\\
Let $x_i$ be node $n_i$.  If $x_i = 1$ then there is a hospital at that node.  If $x_i = 0$ then there is no hospital at that node, but there should be at least one other node $x_j = 1$ and the distance to that node $W(i,j)$ should be between 0 and 1.\\ \\
This is a 0-1 Integer Linear Program given by the following formulation:
\[\begin{array}{rlllllllll}
\min & \sum_{j=1}^{n} (cost_j*node_j) \\
\mathsf{s.t. } \\
& \sum_{j=1}^{n} I(W(i,j) \leq 1)*n_j & \geq 1 \text{ for all }i = 1...n\\
& n_j & \in \{0,1\} \\
\end{array}\]
For the objective function, we want to minimize the cost of placing hospitals.
The constraint
$$
\sum_{j=1}^{n} I(W(i,j) \leq 1)*n_j \geq 1 \text{ for all }i = 1...n
$$
says that for each node $i$ we want to make sure that there is a node $j$ that has a hospital $n_j = 1$ and is within 1 hour of it $I(W(i,j) \leq 1) = 1$. \\

\noindent\textbf{(B)}
\\
The code to the solution is at the end of this document.  The solution itself is the vector 
$$x = [0\quad0\quad0\quad0\quad0\quad1\quad0\quad1]$$
This means in order to minimize the cost of building the hospitals within the driving time of 1 hour for each node, hospitals should be placed at node 6 and node 8.  The total cost of the hospitals ends up being 2.1 million dollars.  

\medskip
\newpage
\noindent\textbf{P4.}
\\
\noindent\textbf{(A)}
We add the indicator variable $z_1...z_n$ to the problem.  If $x_i == 0$ then $z_i = 0$ else $z_i = 1$.  In this problem we want to minimize the sum of $z_i$ and this occurs best when $x_i$ is 0.
\[\begin{array}{rlllllllll}
\min & \sum z \\
\mathsf{s.t. } \\
& Ax & \leq b \\
& x & \leq u*z \\
& x & \geq \ell*z \\
& x & \leq u \\
& x & \geq \ell \\
& z & \in \{0,1\} \\
\end{array}\]
The additional constraints
\begin{align*}
x \leq u*z \\
x \geq \ell*z \\
\end{align*}
say that if $x_i$ is negative then $z_i$ must equal 1, referring to $x \geq \ell*z$.  If $x_i$ is positive then $z_i$ must be equal to 1, referring to $x\leq u*z$.  But if x is 0 then z should be equal to 0 because this is a minimization problem.
\\
\noindent\textbf{(B)}\\
We can use the same problem formulation from above except we make it a maximization problem instead of a minimization.
\[\begin{array}{rlllllllll}
\max & \sum z \\
\mathsf{s.t. } \\
& Ax & \leq b \\
& x & \leq u*z \\
& x & \geq \ell*z \\
& x & \leq u \\
& x & \geq \ell \\
& z & \in \{0,1\} \\
\end{array}\]
The additional constraints
\begin{align*}
x \leq u*z \\
x \geq \ell*z \\
\end{align*}
say that if x is negative then z must equal 1, referring to $x \geq \ell*z$.  If x is positive then z must be equal to 1, referring to $x\leq u*z$.  But if x is 0 then z should be equal to 1 because this is a maximization problem.
\\
\noindent\textbf{(C)}
\\
In order to not satisfy the inequality $a \leq x \leq b$, x needs to be either $\ell \leq x \leq a$ or $b \leq x \leq u$.  The mixed integer program would look like:
\[\begin{array}{rlllllllll}
\max & c^{t}x \\
\mathsf{s.t. } \\
& Ax & \leq b \\
& x \geq \ell*w+(1-w)*b \\
& x \leq a*w+(1-w)*u \\
& w & \in \{1,0\} \\
\end{array}\]
The additional constraints
\begin{align*}
x \geq \ell*w+(1-w)*b \\
x \leq a*w+(1-w)*u \\
\end{align*}
say that if $w_i = 1$ then $\ell \leq x \leq a$.  If $w_i = 0$ then $b \leq x \leq u$.
\newpage
\noindent\textbf{P5.}
\\
We have to find a subset $S$ that contains at least one element in the set $S_i$ for $i = 1,\ldots, k$ and the sum of the elements in $S$ is minimized.
\[\begin{array}{rlllllllll}
\min & \sum_{i=1}^{n} (i*x_i) \\
\mathsf{s.t. } \\
& \sum_{i=1}^{n} (i*x_{ij}) & \geq 1 \text{ for all }j = 1...k\\
& x_i & \in \{0,1\} \\
\end{array}\]
$x_i = 1$ if the element $i$ is in the subset $S$ otherwise $x_i = 0$.  The constraint 
$$
\sum_{j=1}^{n} (j*x_{ji}) \geq 1 \text{ for all }i = 1...k
$$
says that for each set $S_i$, the subset $S$ must contain at least one element from $S_i$.  Solving the above example in the $0-1$ ILP we get the solution vector 
$$
x = [1\quad1\quad0\quad0\quad0\quad0\quad0\quad0\quad0\quad0]
$$ 
This means the subset $S = \{1, 2\}$ which sums to 3.

\lstinputlisting[basicstyle=\scriptsize]{p1a.m}
\lstinputlisting[basicstyle=\scriptsize]{p1b.m}
\lstinputlisting[basicstyle=\scriptsize]{p3b.m}
\lstinputlisting[basicstyle=\scriptsize]{p5.m}

\end{document}
