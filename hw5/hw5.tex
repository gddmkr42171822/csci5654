\documentclass[11pt]{article}
\usepackage{fullpage}
\usepackage{graphicx,enumerate}
\usepackage{amsmath,amssymb,amsfonts}
\renewcommand\vec[1]{\mathbf{#1}}
\usepackage{listings}
\usepackage[table]{xcolor}
\usepackage{framed,comment}
\newcommand\vx {\mathbf{x}}
\newcommand\vy {\mathbf{y}}
\newcommand\vb{\mathbf{b}}
\newcommand\vc{\mathbf{c}}
\newcommand\red[1]{\textcolor{red}{#1}}
\specialcomment{solution}{\bigskip\begin{leftbar}\par\noindent\textbf{Solution.} }{\end{leftbar} }
\excludecomment{solution}
\begin{document}

\begin{tabular}{l}
	\textbf{CSCI 5654-Fall16}: Assignment \#5 (Reading: Slides on ILP, Vanderbei Chapter 23).                                                  \\
	\textbf{Due Date:} Friday, Nov. 4, 2016 (before class)            \\
	\textbf{In-class:} Assignment should be submitted on paper -- no emails. \\
	\textbf{Distance Students:} Assignment may be submitted on paper or
	by email.                                                                \\[10pt]

	\textbf{Your Name:} \phantom{Supercalifragilisticexpialidocius Smith}    \\
	\hline
	                                                                         \\[10pt]
\end{tabular}

\noindent\textbf{P1. (10 points)} Solve each of the problems below using a branch-and-bound method. Write down the solution obtained and the enumeration tree
obtained. You may use any LP solver of your choice to solve the
subproblems. Please do not use a ILP solver directly.

\noindent\textbf{(A)} 
\[\begin{array}{rlllll}
\max & 2 x_1 & - 3 x_2 & -2 x_3 & - x_4 \\
\mathsf{s.t.} & x_1 & - x_2 & + x_3 &   & \leq 5 \\
& x_1 & - 2 x_2 & - x_3 & +x_4   & \leq 3 \\
& x_1 & - x_2 & - x_3 & - x_4 & \leq -1 \\
& x_1, & x_2, & x_3, & x_4 & \in [-5, 5] \\
& x_1, & x_2, & x_3, & x_4 & \in \mathbb{Z} \\
\end{array} \]

\noindent\textbf{Hint:} Use $x_4$ as the variable to branch on.
\\
\begin{tabular}{|l|l|l|}
\hline
Branches & Solution & Optimal value \\ \hline
& $x = [-5\quad-5\quad1.5\quad-.5]$ & -2.5 \\ \hline
$x_4 \leq -1$ & $x = [-3.5\quad-5\quad3.5\quad-1]$ & -2\\ \hline
$x_4 \geq 0$ & $x = [-2\quad-4\quad3\quad0]$ & -2 \\ \hline
$x_4 \leq -1$, $x_3 \leq 3$ & $x = [-4\quad-5\quad3\quad-1]$ & -2 \\ \hline
$x_4 \leq -1$, $x_3 \geq 4$ & $x = [-4\quad-5\quad4\quad-2]$ & -1 \\ \hline
$x_4 \leq -1$, $x_1 \leq -4$ & $x = [-4\quad-5\quad3\quad-1]$ & -2 \\ \hline
$x_4 \leq -1$, $x_1 \geq -3$ & $x = [-3\quad-4.5\quad3.5\quad-1]$ & -1.5 \\ \hline
$x_4 \leq -1$, $x_1 \geq -3$, $x_2 \leq -5$ & Infeasible & \\ \hline
$x_4 \leq -1$, $x_1 \geq -3$, $x_2 \geq -4$ & $x = [-2.5\quad-4\quad3.5\quad-1]$ & -1\\ \hline
$x_4 \leq -1$, $x_1 \geq -3$, $x_3 \leq 3$ & $x = [-3\quad-4\quad3\quad-1]$ & -1\\ \hline
$x_4 \leq -1$, $x_1 \geq -3$, $x_3 \geq 4$ & $x = [-3\quad-4\quad4\quad-2]$ & -1.7764e-15\\ \hline
\end{tabular}

\noindent\textbf{(B)} 
\[ \begin{array}{rlllll}
\max & 2 x_1 &  & + 3 x_3 & + x_4 \\
\mathsf{s.t.}  & x_1 & - x_2 &  & + x_4 & \leq 1 \\
&  & 2 x_2 &  & - x_4 & \leq 2 \\
& x_1 &  & - x_3 & - 2 x_4 & \leq -1 \\
& -x_1 & &   & + x_4 & \leq 1\\
& x_1 &  &   &    & \in \{ -2,-1, 0, 1,2 \} \\
&   & x_2 &  &    & \in \{ -1, 0, 1 \} \\
&   &     & x_3 & & \in \{ 0, 1\} \\
&   &     &    & x_4 & \in \{ -1, 0, 1\} \\
\end{array}\]
\\
\begin{tabular}{|l|l|l|}
\hline
Branches & Solution & Optimal value \\ \hline
& $x = [1.3333\quad1\quad1\quad.6667]$ & 6.3333 \\ \hline
$x_4 \geq 1$ & $x = [1\quad1\quad1\quad1]$ & 6 \\ \hline
$x_4 \leq 0$ & $x = [0\quad1\quad1\quad0]$ & 3 \\ \hline
$x_1 \leq 1$ & $x = [1\quad1\quad1\quad1]$ & 6 \\ \hline
$x_1 \geq 2$ & Infeasible & \\ \hline
\end{tabular}
\\ \\
The optimal value of the objective function is 6.  Two solutions lead to this value:
\begin{align*}
x_1 = 1 \quad x_2 = 1 \quad x_3 = 1 \quad x_4 = 1 \\
x_1 = 0 \quad x_2 = 1 \quad x_3 = 1 \quad x_4 = 0 \\
\end{align*}




\noindent\textbf{P2. (10 points)}  Consider the final dictionaries for 
the LP relaxation of a few ILPs.  Assuming all variables are integers,
write down all the cutting planes:

Dictionary \# 1:
\[\begin{array}{c| c c@{\hskip 2pt} c@{\hskip 2pt} }
 x_{1}   &  0.666666666667 & -0.666667 x_{5} & + 0.333333 x_{4}\\
 x_{2}   &  1 & -1  x_{5} &   \\
 x_{3}   &  2 & + 4  x_{5} & -1  x_{4}\\
\hline
z    &  1 & -1  x_{5} &   \\
\end{array}\]


Dictionary \# 2: 
\[\begin{array}{c| c c@{\hskip 2pt} c@{\hskip 2pt} c@{\hskip 2pt} }
 x_{4}   &  4.33333333333 & + 0.333333 x_{8} & + 0.666667 x_{9} & -0.333333 x_{3}\\
 x_{5}   &  8.66666666667 & -0.333333 x_{8} & + 0.333333 x_{9} & -2.666667 x_{3}\\
 x_{6}   &  10  &    &   & -1  x_{3}\\
 x_{7}   &  3 & -3  x_{8} & + 1  x_{9} & -18  x_{3}\\
 x_{1}   &  5.66666666667 & -0.333333 x_{8} & -0.666667 x_{9} & + 0.333333 x_{3}\\
 x_{2}   &  1.33333333333 & + 0.333333 x_{8} & -0.333333 x_{9} & + 2.666667 x_{3}\\
\hline
z    &  7  &   & -1  x_{9} & -2  x_{3}\\
\end{array}\]

\medskip

\noindent\textbf{P3 (20 points).}  Consider the graph below
which shows various locations and the driving times between them in hours:

\begin{center}
\end{center}

Our goal is to decide whether or not to place a hospital at each node.
The following are the cost of building a hospital at various nodes
in millions of dollars:

\[\begin{tabular}{|c|c|}
\hline
\mbox{Node} & \mbox{Cost}\\
\hline
1 & 3 \\
2 & 3 \\
3 & 1.5 \\
4 & 1 \\
5 & 1.2 \\
6 & 1.3 \\
7 & 0.9 \\
8 & 0.8 \\
\hline
\end{tabular}\]

The following constraints should apply to our placement of hospitals:
each node should either have a hospital or be within $1$ hour driving
distance of a hospital.

For your convenience, the table of pairwise shortest path distances is given 
as an excel spreadsheet.

\noindent\textbf{(A)} Let $G$ be a graph with $n$ nodes
and $W(i,j)$ denote the shortest path weight between nodes $i$ and $j$.
Finally let $\mathbf{c}$ be a $n \times 1$ vector of node costs.
Write down an integer linear programming formulation of the above problem.

\noindent\textbf{(B)} Formulate and solve your ILP for the given example.

\medskip

\noindent\textbf{P4 (15 points)} Consider a polyhedron $P$ given
by the constraints
 \[ A \mathbf{x} \leq \mathbf{b},\ \mathbf{\ell}\ \leq\ \mathbf{x}\ \leq\ \mathbf{u}\,. \]


\begin{enumerate}[(a)]
\item  Write down mixed integer programs that will find the
point $\mathbf{x} \in P$ with the largest number of
$0$ entries in $\mathbf{x}$, 
\item Write down mixed integer programs that will find the
point $\mathbf{x} \in P$ with the smallest number of $0$ entries in $\mathbf{x}$.
\end{enumerate}

\noindent (c) Write down a mixed integer program that will search for a solution
$\mathbf{x} \in P$ maximizing an objective function $\mathbf{c}^t \mathbf{x}$ 
such that $\mathbf{x}$ \emph{does not satisfy}
$\mathbf{a} \leq \mathbf{x} \leq \mathbf{b}$, for given $\mathbf{a}, \mathbf{b}$.

\medskip

\noindent\textbf{P5 (10 points)} We are given  sets of numbers
$\left\langle{S_1,\ldots,S_k}\right\rangle$ such that each $S_i \subseteq \{1,\ldots, n\}$.
For example, $n = 10$ and the sets are
\[ S_1: \{ 1,3,6\},\ S_2:\ \{2, 7, 8\},\ S_3:\ \{1,8,9\},\ S_4:\ \{1,6,5,3\} \,.\]

Our goal is to select a subset $S \subseteq \{ 1, 2, \ldots, n\}$ such that 
$S \cap S_i \not= \emptyset$ for $i = 1,\ldots, k$ and the sum of elements
in the chosen set $S$ is minimized.

Formulate a $0-1$ ILP for the problem for given $n, k, \left\langle{S_1,\ldots,S_k}\right\rangle$. Also,
 solve it for the example above.



\end{document}
