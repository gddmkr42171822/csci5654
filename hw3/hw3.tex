\documentclass[11pt]{article}
\usepackage{fullpage}
\usepackage{amsmath,amssymb,amsfonts}
\renewcommand\vec[1]{\mathbf{#1}}
\usepackage{listings}
\usepackage[table]{xcolor}
\usepackage{framed,comment}
\newcommand\vx {\mathbf{x}}
\newcommand\vy {\mathbf{y}}
\newcommand\red[1]{\textcolor{red}{#1}}
\specialcomment{solution}{\bigskip\begin{leftbar}\par\noindent\textbf{Solution.} }{\end{leftbar} }
\begin{document}

\begin{tabular}{l}
\textbf{CSCI 5654-Fall16}: Assignment \#3 \\
\textbf{Robert Werthman} \phantom{Supercalifragilisticexpialidocius Smith}\\
\hline
\\[10pt]
\end{tabular}

\noindent
\textbf{P1.}
\\
\noindent (A)
\\
Assign dual variables $y_i$ to each of the constraint equations of the primal LP except for the non-negativity constraints.
\[ \begin{array}{rllllll}
\max & 2 x_1  & + 3 x_2 & - x_3 & + x_4 \\
\mathsf{s.t.}\, y_1 \rightarrow & x_1 & - x_2 & + x_3 &   & \leq  -1 \\
y_2 \rightarrow & x_1 &  & + x_3 & + x_4 & \leq 0 \\
y_3 \rightarrow &    & 2 x_2 & + x_3 & + 3 x_4 & \leq  4 \\
& x_1, & x_2, & x_3, & x_4 & \geq  0 \\
\end{array}\]
The dual objective is created from the constants in the $b$ column of the primal LP.  The dual constraints are created by the columns of the primal constraints and the coefficients of the primal objective function.
\[\begin{array}{rllllll}
\min & -y_1 & & +4y_3 \\
\mathsf{s.t.} 
& +y_1 & +y_2 & & \geq 2 \\
& -y_1 & & +2y_3 & \geq 3 \\
& +y_1 & +y_2 & +y_3 & \geq -1 \\
& & +y_2 & +3y_3 & \geq 1 \\
& y_1, & y_2, & y_3 & \geq 0 \\
\end{array}\]
Changing the dual LP into standard form and adding the slack variables $z_1, z_2, \ldots$ we get the following dual problem:
\[\begin{array}{rllllllll}
\max & y_1 & & -4y_3 \\
\mathsf{s.t.} 
& -y_1 & -y_2 & & +z_1 & = -2 \\
& +y_1 & & -2y_3 & +z_2 & = -3 \\
& -y_1 & -y_2 & -y_3 & +z_3 & = +1 \\
& & -y_2 & -3y_3 & +z_4 & = -1 \\
& y_1, & y_2, & y_3, & z_1, & z_2, & z_3, & z_4 & \geq 0 \\
\end{array}\]


\medskip

\noindent (B)
\\
If the primal objective is $\eta = \max -x_1 - x_2 - x_3 - x_4$ then the dual problem in standard form is
\[\begin{array}{rllllllll}
\max & y_1 & & -4y_3 \\
\mathsf{s.t.} 
& -y_1 & -y_2 & & +z_1 & = +1 \\
& +y_1 & & -2y_3 & +z_2 & = +1 \\
& -y_1 & -y_2 & -y_3 & +z_3 & = +1 \\
& & -y_2 & -3y_3 & +z_4 & = +1 \\
& y_1, & y_2, & y_3, & z_1, & z_2, & z_3, & z_4 & \geq 0 \\
\end{array}\]
The dual dictionary is
\[\begin{array}{r|cccccccc}
z_1 & +1 & +y_1 & +y_2 \\
z_2 & +1 & -y_1 & & +2y_3 \\
z_3 & +1 & +y_1 & +y_2 & +y_3 \\
z_4 & +1 & & +y_2 & +3y_3 \\
\hline
-\xi & 0 & +y_1 & & -4y_3
\end{array}\]
Since the dual dictionary is feasible and the primal is not we will apply the simplex method to the dual.  The only entering variable we can choose is $y_1$ and the only leaving variable that constrains $y_1$ is $z_2$.  Solving for $y_1$ in the $z_2$ row and subsituting that equation in for all instances of $y_1$ in the nonbasic variables yields the dictionary
\[\begin{array}{r|cccccccc}
z_1 & +2 & & +y_2  & +2y_3 & & -z_2\\
y_1 & +1 & & & +2y_3 & & -z_2\\
z_3 & +2 & & +y_2 & +3y_3 & & -z_2 \\
z_4 & +1 & & +y_2 & +3y_3 \\
\hline
-\xi & +1 & & & -2y_3 & & -z_2
\end{array}\]
The above dictionary is optimal because there are no entering variables.  We now need to convert the dual dictionary above to the corresponding primal dictionary.  $w_i$ corresponds to $y_i$ and $x_i$ corresponds to $z_i$.  If the variable is nonbasic in the dual dictionary it is basic in the primal dictionary.  The primal dictioary we get is
\[\begin{array}{r|cccccccc}
w_3 & 2 & -2x_1 & -2w_1 & -3x_3 & -3x_4 \\
x_2 & 1 & +x_1 & +1w_1 & +1x_3 & & \\
\hline
\eta & -1 & -2x_1 & -1w_1 & -2x_3 & -1x_4
\end{array}\]
Reinstating the original objective function $\zeta$ fpr $\eta$ we get
\begin{align*}
\zeta & = 2x_1 + 3x_2  - x_3 + x_4 \\
& = 2x_1 + 3(1 +x_1 +w_1 +x_3) - x_3 + x_4 \\
& = 3 + 5x_1 + 2x_3 + x_4 + 3w_1 \\
\end{align*}


\medskip

\noindent (C)
\\
The dual problem is
\[\begin{array}{rllllll}
\min & +y_1 & +y_2 & -4y_3 \\
\mathsf{s.t.}
& +y_1 & +y_2 & -3y_3 & \geq 1 \\
& -2y_1 & +y_2 & & \geq -1 \\
& & -2y_2 & +6y_3 & \geq 1 \\
y_1, & y_2, & y_3 & \geq 0
\end{array}\]
The dual problem in standard form with slack variables $z_1, z_2, \ldots$ and the primal objective function $\eta = -x_1 - x_2 - x_3$ is
\[\begin{array}{rllllll}
\max & -y_1 & -y_2 & +4y_3 \\
\mathsf{s.t.}
& -y_1 & -y_2 & +3y_3 & +z_1 & = +1 \\
& +2y_1 & -y_2 & & +z_2 & = +1 \\
& & +2y_2 & -6y_3 & +z_3 & = +1 \\
y_1, & y_2, & y_3, & z_1, & z_2, & z_3 & \geq 0
\end{array}\]
The dual dictionary is then
\[\begin{array}{r|cccccccc}
z_1 & +1 & +y_1 & +y_2 & -3y_3 \\
z_2 & +1 & -2y_1 & +y_2 & & \\
z_3 & +1 & & -2y_2 & +6y_3 \\
\hline
-\xi & 0 & -y_1 & -y_2 & +4y_3
\end{array}\]
The only entering variable is $y_3$ and the only leaving variable is $z_1$.  This produces the new dual dictionary
\[\begin{array}{r|cccccccc}
y_3 & \frac{1}{3} & +\frac{y_1}{3} & +\frac{y_2}{3} & & -\frac{z_1}{3}\\
z_2 & +1 & -2y_1 & +y_2 & & \\
z_3 & +3 & +2y_1 & & & -2z_1 \\
\hline
-\xi & +\frac{4}{3} & +\frac{y_1}{3} & +\frac{y_2}{3} & & -\frac{4z_1}{3}
\end{array}\]
We see that $y_2$ is an entering variable, but there is no leaving variable for it.  This means this dual dictionary is unbounded and therefore the primal LP is infeasible
\bigskip


\noindent\textbf{P2 (15 points) }  \textbf{(A)} For the LP in problem P1 (A) set up the KKT conditions
following the complementary slackness theorem.
\\
The KKT conditions are
\begin{enumerate}
  \item $Ax + w = b$ (Primal LP feasibility constraints)
  \begin{enumerate}
    \item $x_1 - x_2 + x_3 + w_1 = -1$
    \item $x_1 + x_3 + x_4 + w_2 = 0$
    \item $2x_2 + x_3 +3x_4 + w_3= 4$
    \item $x_1, x_2, x_3, x_4, w_1, w_2, w_3 \geq 0 $
  \end{enumerate}
  \item $A^{t}y - z = c$ (Dual LP feasibility constraints)
  \begin{enumerate}
    \item $y_1 + y_2 - z_1 = 2$
    \item $-y_1 + 2y_3 - z_2 = 3$
    \item $y_1 + y_2 + y_3 - z_3 = -1$
    \item $y_2 + 3y_3 -z_4 = 1$
    \item $y_1, y_2, y_3, z_1, z_2, z_3, z_4 \geq 0$
  \end{enumerate}
  \item $x, y, w, z \geq 0$ (Nonnegativity constraints)
  \item $x_iz_i = 0 \, for \, i = 1,2,\ldots,n$ (Complementary slackness constraints)
  \begin{enumerate}
    \item $x_1z_1 = 0$
    \item $x_2z_2 = 0$
    \item $x_3z_3 = 0$
    \item $x_4z_4 = 0$
  \end{enumerate}
  \item $w_jy_j = 0 \, for \, j = 1,2,\ldots,m$ (Complementary slackness constraints)
  \begin{enumerate}
    \item $w_1y_1 = 0$
    \item $w_2y_2 = 0$
    \item $w_3y_3 = 0$
  \end{enumerate}
\end{enumerate}

\noindent\textbf{(B)} Now solve the KKT conditions to find  dual
variables corresponding to  the following primal feasible solution?
\[ x_1 = 0,\ x_2 = 2,\ x_3 = 0,\ x_4 = 0 \]

Is this primal solution optimal? 

Do not attempt to use an LP solver for this part. Instead substitute
the primal solution into the KKT constraints and solve for dual
solutions that are dual feasible and complementary to the primal
solution above.
\\
First let's find $w_1, w_2, w_3$ with the solution 
\[ x_1 = 0,\ x_2 = 2,\ x_3 = 0,\ x_4 = 0 \]
Plugging that solution into the primal feasibility constraints we get
\[w_1 = 1, w_2 = 0, w_3 = 0\]
Using the the complementary slackness constraints from part A, we get
\[y_1 = 0, z_2 = 0\]
The variables that aren't necessarily 0 are
\[y_2, y_3, z_1, z_3, z_4\]
If we eliminate the variables that are 0 from the dual feasiblity constraints, we are left with the equations
\begin{align*}
y_2 - z_1 &= 2 \\
2y_3 &= 3 \\
y_2 + y_3 - z_3 &= -1 \\
y_2 + 3y_3 - z_4 &= 1 \\
\end{align*}
Guessing values for the variables and solving for those remaining equations, we get the dual solution
\[y_1 = 0, y_2 = 4, y_3 = \frac{3}{2}, z_1 = 2, z_2 = 0, z_3 = \frac{13}{2}, z_4 = \frac{15}{2}\]
If we plug in the primal solution into the primal objective function we get 
$2(0) + 3(2) - (0) + 0 = 6$.  If we plug the dual solution into the dual objective function we get $-(0) + 4(\frac{3}{2}) = 6$.  Because both the dual and primal solution produce the same objective function value we know that the primal solution is optimal.

\bigskip

\noindent\textbf{P3 (10 points)} Consider a standard form LP wherein
two different non-degenerate vertices $\vx_1,\ \vx_2$ are both optimal for the primal. Show using the
complementary slackness theorem that any dual optimal solution
is degenerate.

(\textbf{Hint:} Each vertex saturates precisely $n$ of the
constraints. Now, since the solution $\frac{1}{2} (\vx_1 + \vx_2)$
is also optimal, consider how many constraints this solution
saturates. Apply complementary slackness to derive how many
dual variables should be zero in any dual optimal solution).

\bigskip

\noindent\textbf{P4 (15 points)} \noindent\textbf{(A)} Consider a
simple LP over a hyper-rectangle:

\[ \max\ \mathbf{c}^t \mathbf{x}\ \mathsf{s.t.}\ \ell \leq \mathbf{x}
\leq \mathbf{u} \]

Here $\mathbf{c}$ represents the objective, and $\ell, \mathbf{u}$
represent vectors of upper and lower bounds on $\mathbf{x}$. 

Write down a linear time algorithm for solving the above LP.
(\textbf{Hint:} The solution for each variable $x_i$ can be one of two
possibilities, what are they?)

\medskip

\noindent\textbf{(B)} Using the result in (A) above, write down an
algorithm to solve LPs of the form:

\[ \begin{array}{rl}
\max & \mathbf{c}^t \mathbf{y} \\
\mathsf{s.t.} & \mathbf{y} = A \mathbf{x} \\
& \mathbf{\ell}\ \leq\ \mathbf{x}\ \leq\ \mathbf{u} \\
\end{array}\]

(\textbf{Hint:} Eliminate $\mathbf{y}$ and use the result in (A) ).

\noindent\textbf{P5 (20 points)} We will now prove a well-known
and useful \emph{theorem of the alternative} called Motzkin
transposition theorem using what we have learned so far.

\begin{framed}
\noindent\textbf{Theorem [Motzkin 1936]:} The following system of constraints \textsf{(P)} is infeasible

\[ \begin{array}{l}
A \mathbf{x} \leq \mathbf{b}\\
 \mathbf{x} \geq 0 
\end{array}\]

 if and only if the system of constraints \textsf{(D)}  below is feasible

\[ \begin{array}{l}
A^t \mathbf{y} \geq 0,\\
\mathbf{b}^t \mathbf{y} < 0\\
\mathbf{y} \geq 0 \\
\end{array}\]
\end{framed}

\noindent\textbf{(A)} Prove that if \textsf{(P)} is feasible then
\textsf{(D)} cannot be feasible. (\textbf{Hint:} If $\vx, \vy$ are
simultaneously feasible for \textsf{(P)}, \textsf{(D)} respectively
then derive a contradiction by applying different sets of inequalities
from \textsf{(P)} and \textsf{(D)} to show that $\mathbf{y}^t A \mathbf{x}$ will simultaneously be
$< 0$ and $ \geq 0$.) 

\medskip

\noindent\textbf{(B)} Derive the dual for the auxilliary problem 

\[ \begin{array}{rlll}
\max & \mathbf{0}^t \mathbf{x}  &- x_0 \\
\mathsf{s.t.} & A \mathbf{x} &  - \mathbf{1} x_0 & \leq \mathbf{b} \\
& \mathbf{x},  & x_0 & \geq 0 \\
\end{array}\]
Note that $\mathbf{1}$ is the column vector of all $1$s.
Recall that the auxilliary problem above always has an optimal
solution, and if \textsf{(P)} is infeasible, then the optimal value of this
auxilliary problem is strictly negative. Call it $\beta^* < 0$. 

\medskip

\noindent\textbf{(C)} Prove using \textbf{(B)}  that if system 
\textsf{(P)} is infeasible then system \textsf{(D)} is feasible. (\textbf{Hint:} Use
strong duality to conclude that the dual problem derived in 
\textbf{(B)} has to have an optimal solution whose value
is $\beta^*$. Proceed from there to conclude a solution for system 
(D). )



\end{document}
