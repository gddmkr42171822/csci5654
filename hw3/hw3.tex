\documentclass[11pt]{article}
\usepackage{fullpage}
\usepackage{amsmath,amssymb,amsfonts,listings}
\renewcommand\vec[1]{\mathbf{#1}}
\usepackage{listings}
\usepackage[table]{xcolor}
\usepackage{framed,comment}
\newcommand\vx {\mathbf{x}}
\newcommand\vy {\mathbf{y}}
\newcommand\red[1]{\textcolor{red}{#1}}
\specialcomment{solution}{\bigskip\begin{leftbar}\par\noindent\textbf{Solution.} }{\end{leftbar} }
\begin{document}

\begin{tabular}{l}
\textbf{CSCI 5654-Fall16}: Assignment \#3 \\
\textbf{Robert Werthman} \phantom{Supercalifragilisticexpialidocius Smith}\\
\hline
\\[10pt]
\end{tabular}

\noindent
\textbf{P1.}
\\
\noindent\textbf{(A)}
\\
Assign dual variables $y_i$ to each of the constraint equations of the primal LP except for the non-negativity constraints.
\[ \begin{array}{rllllll}
\max & 2 x_1  & + 3 x_2 & - x_3 & + x_4 \\
\mathsf{s.t.}\, y_1 \rightarrow & x_1 & - x_2 & + x_3 &   & \leq  -1 \\
y_2 \rightarrow & x_1 &  & + x_3 & + x_4 & \leq 0 \\
y_3 \rightarrow &    & 2 x_2 & + x_3 & + 3 x_4 & \leq  4 \\
& x_1, & x_2, & x_3, & x_4 & \geq  0 \\
\end{array}\]
The dual objective is created from the constants in the $b$ column of the primal LP.  The dual constraints are created by the columns of the primal constraints and the coefficients of the primal objective function.
\[\begin{array}{rllllll}
\min & -y_1 & & +4y_3 \\
\mathsf{s.t.} 
& +y_1 & +y_2 & & \geq 2 \\
& -y_1 & & +2y_3 & \geq 3 \\
& +y_1 & +y_2 & +y_3 & \geq -1 \\
& & +y_2 & +3y_3 & \geq 1 \\
& y_1, & y_2, & y_3 & \geq 0 \\
\end{array}\]
Changing the dual LP into standard form and adding the slack variables $z_1, z_2, \ldots$ we get the following dual problem:
\[\begin{array}{rllllllll}
\max & y_1 & & -4y_3 \\
\mathsf{s.t.} 
& -y_1 & -y_2 & & +z_1 & = -2 \\
& +y_1 & & -2y_3 & +z_2 & = -3 \\
& -y_1 & -y_2 & -y_3 & +z_3 & = +1 \\
& & -y_2 & -3y_3 & +z_4 & = -1 \\
& y_1, & y_2, & y_3, & z_1, & z_2, & z_3, & z_4 & \geq 0 \\
\end{array}\]


\medskip

\noindent\textbf{(B)}
\\
If the primal objective is $\eta = \max -x_1 - x_2 - x_3 - x_4$ then the dual problem in standard form is
\[\begin{array}{rllllllll}
\max & y_1 & & -4y_3 \\
\mathsf{s.t.} 
& -y_1 & -y_2 & & +z_1 & = +1 \\
& +y_1 & & -2y_3 & +z_2 & = +1 \\
& -y_1 & -y_2 & -y_3 & +z_3 & = +1 \\
& & -y_2 & -3y_3 & +z_4 & = +1 \\
& y_1, & y_2, & y_3, & z_1, & z_2, & z_3, & z_4 & \geq 0 \\
\end{array}\]
The dual dictionary is
\[\begin{array}{r|cccccccc}
z_1 & +1 & +y_1 & +y_2 \\
z_2 & +1 & -y_1 & & +2y_3 \\
z_3 & +1 & +y_1 & +y_2 & +y_3 \\
z_4 & +1 & & +y_2 & +3y_3 \\
\hline
-\xi & 0 & +y_1 & & -4y_3
\end{array}\]
Since the dual dictionary is feasible and the primal is not we will apply the simplex method to the dual.  The only entering variable we can choose is $y_1$ and the only leaving variable that constrains $y_1$ is $z_2$.  Solving for $y_1$ in the $z_2$ row and subsituting that equation in for all instances of $y_1$ in the nonbasic variables yields the dictionary
\[\begin{array}{r|cccccccc}
z_1 & +2 & & +y_2  & +2y_3 & & -z_2\\
y_1 & +1 & & & +2y_3 & & -z_2\\
z_3 & +2 & & +y_2 & +3y_3 & & -z_2 \\
z_4 & +1 & & +y_2 & +3y_3 \\
\hline
-\xi & +1 & & & -2y_3 & & -z_2
\end{array}\]
The above dictionary is optimal because there are no entering variables.  We now need to convert the dual dictionary above to the corresponding primal dictionary.  $w_i$ corresponds to $y_i$ and $x_i$ corresponds to $z_i$.  If the variable is nonbasic in the dual dictionary it is basic in the primal dictionary.  The primal dictioary we get is
\[\begin{array}{r|cccccccc}
w_3 & 2 & -2x_1 & -2w_1 & -3x_3 & -3x_4 \\
x_2 & 1 & +x_1 & +1w_1 & +1x_3 & & \\
\hline
\eta & -1 & -2x_1 & -1w_1 & -2x_3 & -1x_4
\end{array}\]
Reinstating the original objective function $\zeta$ fpr $\eta$ we get
\begin{align*}
\zeta & = 2x_1 + 3x_2  - x_3 + x_4 \\
& = 2x_1 + 3(1 +x_1 +w_1 +x_3) - x_3 + x_4 \\
& = 3 + 5x_1 + 2x_3 + x_4 + 3w_1 \\
\end{align*}


\medskip

\noindent\textbf{(C)}
\\
The dual problem is
\[\begin{array}{rllllll}
\min & +y_1 & +y_2 & -4y_3 \\
\mathsf{s.t.}
& +y_1 & +y_2 & -3y_3 & \geq 1 \\
& -2y_1 & +y_2 & & \geq -1 \\
& & -2y_2 & +6y_3 & \geq 1 \\
y_1, & y_2, & y_3 & \geq 0
\end{array}\]
The dual problem in standard form with slack variables $z_1, z_2, \ldots$ and the primal objective function $\eta = -x_1 - x_2 - x_3$ is
\[\begin{array}{rllllll}
\max & -y_1 & -y_2 & +4y_3 \\
\mathsf{s.t.}
& -y_1 & -y_2 & +3y_3 & +z_1 & = +1 \\
& +2y_1 & -y_2 & & +z_2 & = +1 \\
& & +2y_2 & -6y_3 & +z_3 & = +1 \\
y_1, & y_2, & y_3, & z_1, & z_2, & z_3 & \geq 0
\end{array}\]
The dual dictionary is then
\[\begin{array}{r|cccccccc}
z_1 & +1 & +y_1 & +y_2 & -3y_3 \\
z_2 & +1 & -2y_1 & +y_2 & & \\
z_3 & +1 & & -2y_2 & +6y_3 \\
\hline
-\xi & 0 & -y_1 & -y_2 & +4y_3
\end{array}\]
The only entering variable is $y_3$ and the only leaving variable is $z_1$.  This produces the new dual dictionary
\[\begin{array}{r|cccccccc}
y_3 & \frac{1}{3} & +\frac{y_1}{3} & +\frac{y_2}{3} & & -\frac{z_1}{3}\\
z_2 & +1 & -2y_1 & +y_2 & & \\
z_3 & +3 & +2y_1 & & & -2z_1 \\
\hline
-\xi & +\frac{4}{3} & +\frac{y_1}{3} & +\frac{y_2}{3} & & -\frac{4z_1}{3}
\end{array}\]
We see that $y_2$ is an entering variable, but there is no leaving variable for it.  This means this dual dictionary is unbounded and therefore the primal LP is infeasible
\bigskip


\noindent\textbf{P2.}
\\
\noindent\textbf{(A)}
\\
The KKT conditions are
\begin{enumerate}
  \item $Ax + w = b$ (Primal LP feasibility constraints)
  \begin{enumerate}
    \item $x_1 - x_2 + x_3 + w_1 = -1$
    \item $x_1 + x_3 + x_4 + w_2 = 0$
    \item $2x_2 + x_3 +3x_4 + w_3= 4$
    \item $x_1, x_2, x_3, x_4, w_1, w_2, w_3 \geq 0 $
  \end{enumerate}
  \item $A^{t}y - z = c$ (Dual LP feasibility constraints)
  \begin{enumerate}
    \item $y_1 + y_2 - z_1 = 2$
    \item $-y_1 + 2y_3 - z_2 = 3$
    \item $y_1 + y_2 + y_3 - z_3 = -1$
    \item $y_2 + 3y_3 -z_4 = 1$
    \item $y_1, y_2, y_3, z_1, z_2, z_3, z_4 \geq 0$
  \end{enumerate}
  \item $x, y, w, z \geq 0$ (Nonnegativity constraints)
  \item $x_iz_i = 0 \, for \, i = 1,2,\ldots,n$ (Complementary slackness constraints)
  \begin{enumerate}
    \item $x_1z_1 = 0$
    \item $x_2z_2 = 0$
    \item $x_3z_3 = 0$
    \item $x_4z_4 = 0$
  \end{enumerate}
  \item $w_jy_j = 0 \, for \, j = 1,2,\ldots,m$ (Complementary slackness constraints)
  \begin{enumerate}
    \item $w_1y_1 = 0$
    \item $w_2y_2 = 0$
    \item $w_3y_3 = 0$
  \end{enumerate}
\end{enumerate}

\noindent\textbf{(B)}
\\
First let's find $w_1, w_2, w_3$ with the solution 
\[ x_1 = 0,\ x_2 = 2,\ x_3 = 0,\ x_4 = 0 \]
Plugging that solution into the primal feasibility constraints we get
\[w_1 = 1, w_2 = 0, w_3 = 0\]
Using the the complementary slackness constraints from part A, we get
\[y_1 = 0, z_2 = 0\]
The variables that aren't necessarily 0 are
\[y_2, y_3, z_1, z_3, z_4\]
If we eliminate the variables that are 0 from the dual feasiblity constraints, we are left with the equations
\begin{align*}
y_2 - z_1 &= 2 \\
2y_3 &= 3 \\
y_2 + y_3 - z_3 &= -1 \\
y_2 + 3y_3 - z_4 &= 1 \\
\end{align*}
Guessing values for the variables and solving for those remaining equations, we get the dual solution
\[y_1 = 0, y_2 = 4, y_3 = \frac{3}{2}, z_1 = 2, z_2 = 0, z_3 = \frac{13}{2}, z_4 = \frac{15}{2}\]
If we plug in the primal solution into the primal objective function we get 
$2(0) + 3(2) - (0) + 0 = 6$.  If we plug the dual solution into the dual objective function we get $-(0) + 4(\frac{3}{2}) = 6$.  Because both the dual and primal solution produce the same objective function value we know that the primal solution is optimal.

\bigskip

\noindent\textbf{P3.}
\\
For the solution $\frac{1}{2} (\vx_1 + \vx_2)$ it is possible that n-1 constraints are saturated.  For example, if you have a problem with 2 variables (2 dimensional space) and 4 constraints, and it is in the shape of a square, it is possible for two optimal solutions ($x_1$, $x_2$) to be vertices on a single line that are satured by 2 lines.  The solution $\frac{1}{2} (\vx_1 + \vx_2)$ would be the midpoint on that single line and only be saturated by 1 constraint.\\
\\
If n-1 constraints are saturated in the primal then n-1 variables are non-basic in the primal.  The Complementary Slackness Theorem says there will be n+m variables, between the dual and the primal, that are equal to 0.  This means there should be n+m saturated constraints.  If n-1 variables are non-basic in the primal, then there are m+1 variables equal to 0 in the dual.  This means there are m+1 saturated constraints in the dual.  Given that there can only be m non-basic variables in the dual, there is one basic variable in the dual equal to 0.  This means the the dual optimal solution, when there there are two optimal, non-degenerate solutions to the primal, is degenerate.   

\bigskip

\noindent\textbf{P4.}
\\
\noindent\textbf{(A)}
\\
\subsection*{Psuedocode}
{\fontfamily{pcr}\selectfont
FindMaxA($c$, $x$, $u$, $l$) \{\\
$\zeta = 0$ \# Objective function value\\
$i = 0$ \\
while $i$ < size of $c$:\\
\hspace*{1em} if $c_i$ < 0: \\
\hspace*{2em} $x_i = l_i$ \\
\hspace*{1em} else: \\
\hspace*{2em} $x_i = u_i$ \\
\hspace*{1em} $\zeta = \zeta + (c_i*x_i)$ \\
\hspace*{1em} $i = i + 1$ \\
return $\zeta$, $x$ \\
\}
}
\subsection*{Correctness Proof}
We wish to maximize the objective function $c^tx$ of the linear program.  $x$'s lower bound is $l$ and its upper bound is $u$.  It can take on either of these values.  Because we want to maximize the objective function, if $c_i < 0$ -- $c_i$ is negative -- we want $x_i$ to be the lowest possible value $l$ so when $c_i*x_i$ is added to the objective function value $\zeta$, $\zeta$ decreases as little as possible.  If $c_i \ge 0$ -- $c_i$ is positive or 0 -- we want $x_i$ to be the largest possible value $u$ so when $c_i*x_i$ is added to the objective function value $\zeta$, $\zeta$ increases as much as possible.
\subsection*{Runtime Proof}
We iterate through the vector $c$ which is of size $n$ for the number of variables in the vector $x$.  We do $n$ variable assignments and $n$ additions.  The runtime is O(n).

\medskip

\noindent\textbf{(B)}
\\
\subsection*{Psuedocode}
{\fontfamily{pcr}\selectfont
FindMaxB($c$, $A$, $x$, $l$, $u$) \{ \\
\hspace*{1em} z = column vector of size n with all 0's \\
\hspace*{1em} m = number of rows in A \\
\hspace*{1em} n = number of columns in A \\
\hspace*{1em} for(j=0;j<n;i++) \{ \\
\hspace*{2em} for(i=0;i<m;i++) \{ \\
\hspace*{3em} z[j] = z[j] + c[i]*A[i][j] \\
\hspace*{2em} \} \\
\hspace*{1em} \} \\
\hspace*{1em} $\zeta$, $x$ = FindMaxA(z, x, u, l) \\
\hspace*{1em} return $\zeta$, $x$ \\
\}
}
\subsection*{Correctness Proof}
We first combine the constants in the c vector and A matrix into another vector z.  The values in z will now either be positive or negative.  We use the function FindMaxA() from part (A) to determine which value, l or u, x should be set to.  We wish to maximize the objective function so if the value in z is negative we want to choose the smallest value of x.  If z is positive we want to choose the largest value of x.  This will ensure the objective function value $\zeta$ is maximized.  
\subsection*{Runtime Proof}
We go through each row and column of the matrix A.  This takes m*n time.  When then iterate through c in FindMaxA() from part (A) and this takes n times.  The runtime is then O(m*n+n).

\bigskip

\noindent\textbf{P5.}
\\
\noindent\textbf{(A)}
\\
We need to show $y^tAx$ can be $<0$ and $\geq 0$ simultaneously.\\
\\
Because $y \geq 0$
\begin{align*}
&Ax \leq b \\
&\Rightarrow y^tAx \leq y^tb \\
&= b^ty < 0
\end{align*}
This shows $y^tAx < 0$.\\
\\
Because $x \geq 0$
\begin{align*}
&A^ty \geq 0 \\
&\Rightarrow x^tA^ty \geq x^t*0\\
&\Rightarrow (Ax)^ty \geq 0 \\
&\Rightarrow ((Ax)^ty)^t \geq 0 \\
&\Rightarrow y^tAx \geq 0
\end{align*}
This shows $y^tAx \geq 0$.\\
\\ 
If both systems are feasible then $y^tAx \geq 0$ and $y^tAx < 0$ which is impossible.  This means only one system can be feasible at a time. 
\medskip

\noindent\textbf{(B)}
\\
The dual problem of the auxillary problem is
\[ \begin{array}{rlll}
\min & b^ty \\
\mathsf{s.t.} 
& A^ty &  & \geq 0 \\
& -1^ty & & \geq -1 \\
& y & & \geq 0
\end{array}\]
The optimal value of the auxillary problem is $\beta^*$ which is $< 0$.

\medskip

\noindent\textbf{(C)}
\\
Strong duality theorem says
\[
c^tx = b^ty
\]
We assume system (P) is infeasible and system (D) is feasible.  We then have the system (D) constraints
\[ \begin{array}{l}
A^t \mathbf{y} \geq 0,\\
\mathbf{b}^t \mathbf{y} < 0\\
\mathbf{y} \geq 0 \\
\end{array}\]
that have to be satisfied.  If we assume strong duality is true in this case then the dual problem objective function value of the auxiliary problem equals the auxiliary problem objective function value.
\[
b^ty = c^tx - x_0 = \beta* < 0
\]
This means the constraint $b^ty < 0$ of system (D) is satisfied.  This also means all of the the other contraints of system (D) match the constraints of the dual problem of the auxiliary problem except for $-1^ty \geq -1$.  In this case, because $b^ty < \beta*$ this constraints is always satisfied so it can be eliminated from the problem.



\end{document}
